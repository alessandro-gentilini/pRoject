%************************************************
\section{Text manipulation}\label{sec: text}
%************************************************
\samp{install.package(stringr)}\\
\samp{library(stringr)}
\bigskip 

Use \samp{str\_trim} to trim leading and 
tailing white spaces:
\begin{example}
s <- ' Hello, world! '

R: str_trim(s, side = "left") | R: str_trim(s, side = "right")
[1] "Hello, world! "          | [1] " Hello, world!"

R: str_trim(s)                | R: str_trim("\n\nHello, world!\t")
[1] "Hello, world!"           | [1] "Hello, world!"
\end{example}
In order to replace \emph{all} white space (and likewise
any other character) use \samp{str\_replace\_all}.
\begin{example}
R: str_replace_all(s, fixed(" "), "") 

[1] "Hello,world!"

R: str_replace_all(s, "l", "!")
[1] " He!!o, wor!d! "
\end{example}
The functions \samp{tolower} and \samp{toupper}
do the job as named:
\begin{example}
R: tolower(s)         | R: toupper(s)
[1] "hello, world!"   | [1] "HELLO, WORLD!"
\end{example}
Strings can be alphabetically sorted using the 
\samp{sort} numerical function plus a little manipulation
of the characters. This can be useful when checking
whether a certain number of words having the same number
of characters are anagrams of one other: the standars
procedure is to split their letters and sort them
alphabetically to match them. 
\begin{example}
sort.word <- function(x){
 x <- tolower(x)
 x <- str_replace_all(x, fixed(" "), "")    
 x <- paste(sort(unlist(strsplit(x, ""))), collapse = "") 
 return(x)
}

is.anagram <- function(x,y){
    return(sort.word(x) == sort.word(y))
}

first  <- "Eleven plus Two"
second <- "Twelve plus One"
third  <- "Ten plus three"

R: is.anagram(first, second)    | R: is.anagram(first, third)
[1] TRUE                        | [1] FALSE                        
\end{example}


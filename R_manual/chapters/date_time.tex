%************************************************
\section{Date and time formats: lubridate}\label{sec: time}
%************************************************
The function \samp{as.Date} converts the most
date formats given to \emph{input} them 
into the rules of the ISO-8601 
international standard, which expresses the dates
as year-month-day. The format to be converted 
must correspond to the introduced date format:
\begin{example}
date1 <- as.Date("19/02/87", format = "%d/%m/%y")
date2 <- as.Date("04-06-15", format = "%d-%m-%y")

R: date1
[1] "1987-02-19"

R: date1 > date2
[1] FALSE

R: year(date1)
[1] 1987

R: week(date1)
[1] 8
\end{example}

Different placeholders after the percentage sign $\%$
correspond to different date formats. A full list is
available here\footnote{
\url{https://stat.ethz.ch/R-manual/R-patched/library/base/html/strptime.html}}.
Also, the function \samp{strptime} converts between character 
representations and objects obigf classes "POSIXlt" 
and "POSIXct" representing calendar dates and times;
consequently, it is used to \emph{output} a given date in a 
different desired time format or representation.
\medskip 
The functions \samp{as.POSIXct} and \samp{as.POSIXlt}
give representation in the central (local, respectively)
time stamp format as
\begin{example}
R: as.POSIXct(Sys.Date())
[1] "2015-10-17 02:00:00 CEST"

R: as.POSIXlt(Sys.Date())
[1] "2015-10-17 UTC" 
\end{example}
\bigskip

The package \samp{lubridate} simplifies the date and
time arithmetics as
\begin{example}
install.package(lubridate)
library(lubridate)

R: date1  + weeks(5)
[1] "1987-03-26"

R: date1 - years(2)
[1] "1985-02-19"
\end{example}
Also notice the additional functions giving back
precise information on the weekday and position in
the year as \samp{ymd\_hms} or 
\begin{example}
R: wday(Sys.Date())
[1] 7

R: wday(Sys.Date(), label = TRUE)
[1] Sat 
\end{example}
and the function \samp{isoweek}
\begin{example}
date1 <- as.Date("2014-12-31")

R: isoweek(date1)
[1] 1
R: week(date1)
[1] 53
\end{example}

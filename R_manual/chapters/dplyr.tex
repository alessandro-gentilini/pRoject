%************************************************
\section{The \pkg{dplyr} package}\label{sec: dplyr}
%************************************************
\samp{install.packages(`dplyr')}\\
\samp{library(`dplyr')}
\bigskip 

\samp{dplyr} allows pretty much the same operations as
\samp{data.table}, only with a different grammar. We
will exploit its features showing the equivalent
\samp{data.table} syntax for comparison.

\subsection{Subsetting a data set upon constraints}
Data sets can be ordered by columns values as
follows:
\begin{example}
		dplyr
R: arrange(iris, Species, desc(Sepal.Length))
 
 		data.table
R: setorder(iris, Species, -Sepal.Length)
 
   Sepal.Length Sepal.Width Petal.Length Petal.Width Species
1          5.8         4.0          1.2         0.2  setosa
2          5.7         4.4          1.5         0.4  setosa
3          5.7         3.8          1.7         0.3  setosa
4          5.5         4.2          1.4         0.2  setosa
5          5.5         3.5          1.3         0.2  setosa
\end{example}

Columns in a data set can be accessed by name
or position reference:
\begin{example}
		dplyr
R: select(iris, Species , Petal.Width, Petal.Length)
R: select(iris, c(5,4,3))
 
		data.table
R: iris[, .(Species, Petal.Width, Petal.Length)]
R: iris[, c(5,4,3), with = FALSE]

   Species Petal.Width Petal.Length
1  setosa         0.2          1.4
2  setosa         0.2          1.4
3  setosa         0.2          1.3
4  setosa         0.2          1.5
5  setosa         0.2          1.4
\end{example}
New columns can be defined and added as
\begin{example}
		dplyr
R: mutate(iris, new_value = Sepal.Length/Sepal.Width)
R: select(iris, new_values, Species)

# to only keep the new variables use
R: transmute(iris, new_value = Sepal.Length/Sepal.Width, Species)

		data.table
R: iris[, .(new_value = Sepal.Length/Sepal.Width, Species)]

     new_value   Species
  1:  1.457143    setosa
  2:  1.633333    setosa
  3:  1.468750    setosa
  4:  1.483871    setosa
  5:  1.388889    setosa
\end{example}
and can be deleted as 
\samp{select(iris, -Petal.Width, -Petal.Length)}.
\bigskip

Rows can be subset according to constraints
\begin{example}
		dplyr
R: filter(iris, 
       Species == 'virginica'
       & (Petal.Width > 2.3 | Sepal.Width <3))

R: filter(iris,
          !Species %in% c("virginica", "setosa"))
       
		data.table
R: subset(iris, 
       Species == 'virginica'
       & (Petal.Width > 2.3 | Sepal.Width < 3)) 
       
R: subset(iris,
          !Species %in% c("virginica", "setosa"))       
\end{example}
The rows in correspondence of a match
can be extracted with
\begin{example}
		dplyr
R: iris %>% 
     group_by(Species)  %>% 
       filter(Petal.Length == max(Petal.Length))
       
		data.table
R: iris[
        iris[, 
             .I[Petal.Length = max(Petal.Length)], 
	     by = Species]$V1
       ]
	
   Sepal.Length Sepal.Width Petal.Length Petal.Width    Species
1:          5.1         3.5          1.4         0.2     setosa
2:          6.5         2.8          4.6         1.5 versicolor
3:          7.6         3.0          6.6         2.1  virginica
\end{example}

Simple frequencies counts per group can be obtained 
via the \samp{count} function
\begin{example}
		dplyr
R: iris %>% 
     count(Species) 

		data.table
R: iris[, 
        .N,
        by = Species
        ]

     Species  n
1     setosa 50
2 versicolor 50
3  virginica 50
\end{example}
The equivalent of \samp{lapply(.SD, fun)} is:
\begin{example}
		dplyr
R: iris %>%
     group_by(Species) %>%
       summarise_each(funs(sd))
 
		data.table
R: iris[,
	lapply(.SD, sd),
	by = Species
	]

     Species Sepal.Length Sepal.Width Petal.Length Petal.Width
1     setosa    0.3524897   0.3790644    0.1736640   0.1053856
2 versicolor    0.5161711   0.3137983    0.4699110   0.1977527
3  virginica    0.6358796   0.3224966    0.5518947   0.2746501
\end{example}

\subsection{Random and unique rows}
Random rows can be easily subset with
\begin{example}
		dplyr
R: set.seed(1)
R: sample_n(quarks, 10)

		data.table
R: set.seed(1)
R: quarks[sample(.N,10)]
\end{example}
Unlike \samp{data.table}, the package 
\samp{dplyr} allows for sampling in percentage
as fractions of the number of elements of the 
initial data set: \samp{sample\_frac(df, size = 0.3)}
does the job, for example.
Unique values can be fetched on 
constraints as
\begin{example}
		dplyr 
R: quarks %>%
     distinct(flavour, S_z)
     
		data.table
R: unique(quarks, by = c("flavour", "S_z"))
\end{example}

\subsection{Grouping by}
Variables can by grouped by according to
the following grammar:
\begin{example}
		dplyr
R: iris %>% 
    group_by(Species) %>%
        summarise(mean = mean(Petal.Width), dev = sd(Petal.Width))
        
R: iris[,
        .(mean = mean(Petal.Width), dev = sd(Petal.Width)), 
        by = Species
        ]

      Species  mean       dev
1:     setosa 0.246 0.1053856
2: versicolor 1.326 0.1977527
3:  virginica 2.026 0.2746501

\end{example}


\subsection{Joining data sets}
Joins in \samp{dplyr} can be performed using
straightforward although verbose syntax, as 
shown in the following.
\bigskip 

\subsubsection*{Inner joins}
Given two data sets with at least one common
variable, inner joins are performed using 
the following expression:
\begin{example}
set.seed(10)
first <- sample_n(quarks,10)

set.seed(20)
second<- sample_n(quarks,10)

R: inner_join(first, second, by = c("lab", "flavour"))

  lab flavour S_z.x S_z.y
1   B strange   1/2  -1/2
2   C strange  -1/2   1/2
3   C strange   1/2   1/2
4   C strange   1/2   1/2
5   D  bottom   1/2   1/2
6   F  bottom   1/2   1/2
\end{example}

\subsubsection*{Left joins}
Equivalently for left joins
\begin{example}
R: left_join(first, second, by = c("lab", "flavour")) 

   lab flavour S_z.x S_z.y
1    B  charme   1/2    NA
2    B strange   1/2  -1/2
3    C  charme  -1/2    NA
4    C strange  -1/2   1/2
5    C strange   1/2   1/2
6    C strange   1/2   1/2
7    D  bottom   1/2   1/2
8    D strange  -1/2    NA
9    E  charme  -1/2    NA
10   F  bottom   1/2   1/2 
\end{example}
Unlike the \samp{X[Y]} method in \samp{data.table},
columns are here not automatically sorted after 
having been merged.

\subsubsection*{Full joins}
Along the same lines:
\begin{example}
R: full_join(first, second, by = c("lab", "flavour"))

   lab flavour S_z.x S_z.y
1    C strange  -1/2   1/2
2    C strange   1/2   1/2
3    D strange  -1/2  <NA>
4    E  charme  -1/2  <NA>
5    D  bottom   1/2   1/2
6    B  charme   1/2  <NA>
7    C  charme  -1/2  <NA>
...
\end{example}

\subsubsection*{Anti-joins}
The anti-joins returns all the rows in the first
data sets not present in the second one
\begin{example}
R: anti_join(first, second, by = c("lab", "flavour"))

  lab flavour  S_z
1   B  charme  1/2
2   C  charme -1/2
3   D strange -1/2
4   E  charme -1/2
\end{example}

